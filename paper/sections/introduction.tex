\section{Introduction}
\label{introduction}

Articular cartilage is flexible connective tissue that is organised in
a manner so as to permit stability and movement of the skeleton. This
connective tissue consists of an extra-cellular matrix (composed
primarily of collagen, elastin and proteoglycans) and one type of
cell---the {\em chondrocyte}---which is responsible for synthesis and
homeostasis of the matrix. Under abnormal conditions, the balance
between matrix synthesis and degradation is lost, causing inflammation
of the tissue and/or osteoarthritis, which is the wearing out of the
cartilage layer causing painful bone against bone friction.

The goal of this study is to understand what provokes these changes in
matrix turnover by modelling the cellular physiology of the
chondrocyte. In this paper, an electrophysiological model is presented
to study the influence electro-chemical environment on the behaviour
of the chondrocyte. To our knowledge, this is the first model of its
kind. The model is used to study the effect of the local pH on the
chondrocyte. Further extensions to a multi-cellular model, including
accounting for the effect of mechanical stresses will follow in
subsequent articles.

% Local Variables:
% TeX-master: "chondrocyte-model"
% mode: latex
% mode: flyspell
% End:
