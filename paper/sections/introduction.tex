\section*{Introduction}
\label{introduction}

\todo{{\bf Harish:} Add remaining citations to the bibtex file.}

Articular cartilage is aneural, avascular, alymphatic, flexible
connective tissue that covers the articulating ends of diarthroidal
joints \citep{Poole1997,Mankin1982} and permits stability and movement
of the skeleton. This connective tissue consists of an extracellular
matrix (ECM, composed primarily of collagen, elastin and
proteoglycans, as detailed below) and one type of cell---the {\em
  chondrocyte}---which is responsible for synthesis and homeostasis of
the matrix. Articular cartilage is regularly exposed to mechanical
stresses, and this exposure is essential for the health of the tissue
\citep{Stockwell1991}. Chondrocytes occupy only 1--10\% of the total
volume of articular cartilage in mammals \citep{CarneyMuir1988,
  Halletal1996} and play no direct mechanical role. Instead,
mechanical support is provided by the ECM, which is composed of (a)
collagen fibers, which gives the tissue the ability to resist tension,
(b) negatively-charged gel-like proteoglycans (PGs) trapped within the
collagen mesh, allowing the tissue to bear compression
\citep{Poole1997, BuckwalterMankin1998} and (c) synovial fluid within
the articular capsule which acts as a lubricant, allowing for free
movement of the bones \citep{Edwardsetal1994}. The chondrocyte thus
resides in a physiologically atypical and dynamic environment and its
primary role is to maintain viable cartilage by balancing
macromolecular synthesis and breakdown (see
e.g. \citet{Wilkinsetal2000, Stockwell1991, Fassbender1987}).

Under abnormal conditions, chondrocyte damage may occur and the
balance between matrix synthesis and degradation is lost, causing
inflammation of the tissue and/or osteoarthritis: a wearing out of the
cartilage layer which causes painful, bone-against-bone friction. It
is generally known that the progression of osteoarthritis (Rush and
Hall, 2003) and limited recovery of chondrocytes (Jones et al, 1999)
is linked to poorly-regulated volume changes \citep{Lewisetal2011};
physical damage to cartilage is easier in the context of reduced
osmolarity (Bushet et al, 2005).  In turn, there is indication that
these volume changes are linked to abnormal maintenance of resting
membrane potential in these cells \citep{Lewisetal2011}.  In abnormal
cells, the response to challenging external stimuli may be altered
(e.g. much larger changes in resting membrane potential) as compared
to healthy cells (Lewis et al, 2011; WIlson, et al 2004; Tsuga 2002;
Tirabashi 2010a).  It has been suggested that such changes in the
regulation of the resting membrane potential are due to altered ion
channel function (Lewis et al, 2011; WIlson, et al 2004; Tsuga 2002;
Tirabashi 2010a).  Direct experimental investigation of the link
between chondrocyte electrophysiology and chondrotoxicity is
complicated, however, by chondrocytes' small cell size and the
associated limitations of in vitro electrophysiological studies. We
have thus developed and present here a detailed, biophysically-based
model of chondrocyte electrophysiology.  The first of its kind, the
model will facilitate investigation of questions related to
chondrocyte electrophysiology and signaling, and provides a profound
basis for subsequent models of chondrocyte and articular
pathophysiology.

Osteoarthritic changes may develop in even young patients following
orthopedic surgery (cite) via chondrolysis, a condition in which
accelerated loss of articular cartilage occurs over a short time
period (Webb editorial, 2009).  Several clinical studies have
suggested that this significant chondrotoxicity can occur as a result
of postoperative administration of bupivacaine, a local anesthetic
(Busfield and Romero, 2009; Bailie and Ellenbecker, 2009; Rapley et
al, 2009; Wiater, et al, 2011).  Experimental work has confirmed that
bupivacaine reveals profound chondrotoxic effects in both cell
\citep{Chuetal2006} and animal studies (Gomoll et al, 2006; Chu et al,
2010).  The exact mechanisms leading to chondrotoxicity in this
context remain unclear.  However, it appears that the mechanism of
toxicity is unrelated to the primary mechanism of action of
bupivacaine, the blockade of voltage-gated sodium channels (cite), and
instead may be related to potassium channel blockade (Grishko, et al,
2010).  Additionally, it is known that the family of two-pore K+
channels, recently identified in human articular chondrocytes (Clark,
et al, manuscript) is likely affected by bupivacaine administration
(Clark et al, manuscript; Punke et al 2003).  We therefore hypothesize
that the blockade of the two-pore K+ channel in human articular
chondrocytes by the local anesthetic bupivacaine leads to abnormal
regulation of the resting membrane potential in these cells, which may
concomitantly lead to abnormal volume regulation, altered signaling,
and cell death.  This paper aims to (1) present the first model of
chondrocyte electrophysiology in detail and, (2) to use this model to
investigate the potential role of bupivacaine in the homeostasis of
the chondrocyte resting membrane potential and subsequent volumetric
changes.

% Local Variables:
% TeX-master: "chondrocyte-model"
% mode: latex
% mode: flyspell
% End:
