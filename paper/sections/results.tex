\section*{Results}
\label{results}

%% \item The characteristics of the model
%% \item The results of testing our hypothesis, namely that cupivacaine
%%   blocks $I_2poreK$, leading to changes in the RMP, leading to changes
%%   in the volume regulation (which concomitantly leads to changes in
%%   signaling and potentially apoptosis, or changes therein)
%% \end{itemize}

\subsection*{Model characteristics}

Experimental conditions described in \citet{Clarketal2011} were first
replicated in the simulations to fit significant potassium currents
used in the model. For this, the exterior concentrations of the
various species in the model were set to experimental values (\Ko =
5~mM, \Nao = 140~mM, \Cao = 2~mM, pH = 7.4 for $I_{\rm K_{Ca-act}}$
and $I_{\rm K_{ur}}$) and (\Ko = 5~mM, \Nao = 145~mM, \Cao = 2~mM, pH
= 8.5 for $I_{\rm K_{2\, pore}}$), and the potential was linearly ramped
over 1~s from -130~mV to +100~mV. The evolution of the concentrations
and currents were tracked in this period, and multiple simulations
of this nature were used to fit parameters for $I_{\rm K_{2\, pore}}$,
$I_{\rm K_{Ca-act}}$ and $I_{\rm
  K_{ur}}$). Figure~\ref{fig:potassium-currents} shows these
individual currents and their fit to experimental values from
\citet{Clarketal2011}. The other currents used in the model are
significantly smaller in magnitude, and have not been specifically fit
due to lack of experimental data. (Figure~\ref{fig:other-currents})

After this primary parameteristion, the overall model behaviour was
studied under a linear voltage ramp from -130~mV to +90~mV under
conditions matching experiments (\Ko = 5~mM, \Nao = 140~mM, \Cao =
2~mM, pH = 7.4) to reveal that the overall voltage-current behaviour
of the chondrocyte model reproduces the experimental data quite
closely \citep{Clarketal2011}. This comparison is shown in
Figure~\ref{fig:overall-behaviour} along with corresponding time
evolution of the total current in the model.

The initial conditions used in the simulations were steady
state values of the solution under same conditions used for the
different numerical experiments. Figure~\ref{fig:concentrations} shows
that, when starting from a steady state solution, the concentrations
in the model do not evolve over a relatively long simulation period of
30~min. The initial conditions for the concentrations used in the
computations were \Nai = 2.814~mM, \Ki = 121.59~mM, \Cai =
2.371e-06~mM, \Hi = 6.188e-10~mM, \Cli = 13.209~mM. When the model is
perturbed from these conditions, it returns to steady state values in
a similar time frame.

\subsection*{Bupivicane as a volume regulator in the chondrocyte}

% Local Variables:
% TeX-master: "chondrocyte-model"
% mode: latex
% mode: flyspell
% End:
