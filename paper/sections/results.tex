\section*{Results}
\label{results}

\subsection*{Model characteristics}

Experimental conditions described in \citet{Clarketal2011} were first
replicated in the simulations to fit significant potassium currents
used in the model. For this, the exterior concentrations of the
various species in the model were set to experimental values (\Ko =
5~mM, \Nao = 140~mM, \Cao = 2~mM, pH = 7.4 for $I_{\rm K_{Ca-act}}$
and $I_{\rm K_{ur}}$) and (\Ko = 5~mM, \Nao = 145~mM, \Cao = 2~mM, pH
= 8.5 for $I_{\rm K_{2\, pore}}$), and the potential was linearly ramped
over 1~s from -130~mV to +100~mV. The evolution of the concentrations
and currents were tracked in this period, and multiple simulations
of this nature were used to fit parameters for $I_{\rm K_{2\, pore}}$,
$I_{\rm K_{Ca-act}}$ and $I_{\rm
  K_{ur}}$). Figure~\ref{fig:potassium-currents} shows these
individual currents and their fit to experimental values from
\citet{Clarketal2011}. The other currents used in the model are
significantly smaller in magnitude, and have not been specifically fit
due to lack of experimental data. (Figure~\ref{fig:other-currents})

After this primary parameteristion, the overall model behaviour was
studied under a linear voltage ramp from -130~mV to +90~mV under
conditions matching experiments (\Ko = 5~mM, \Nao = 140~mM, \Cao =
2~mM, pH = 7.4) to reveal that the overall voltage-current behaviour
of the chondrocyte model reproduces the experimental data quite
closely \citep{Clarketal2011}. This comparison is shown in
Figure~\ref{fig:overall-behaviour} along with corresponding time
evolution of the total current in the model.

The initial conditions used in the simulations were steady
state values of the solution under same conditions used for the
different numerical experiments. Figure~\ref{fig:concentrations} shows
that, when starting from a steady state solution, the concentrations
in the model do not evolve over a relatively long simulation period of
30~min. The initial conditions for the concentrations used in the
computations were \Nai = 2.814~mM, \Ki = 121.59~mM, \Cai =
2.371e-06~mM, \Hi = 6.188e-10~mM, \Cli = 13.209~mM. When the model is
perturbed from these conditions, it returns to steady state values in
a similar time frame.

\subsection*{Bupivicane as a volume regulator in the chondrocyte}

With the basic model behaviour tuned and validated with respect to
experimental data, we turn to study the effects of blocking $I_{\rm
  K_{2\, pore}}$ with bupivicane. We hypothesise that the blockage of
this current results in an increase in resting membrane potential
(RMP) of the cell, which limits its ability to regulate its
volume. Several computations were carried out with different
fractions of $I_{\rm K_{2\, pore}}$ blocked, from 0\% to 100\%, and
the model was allowed to freely evolve until it reached steady state
(usually in the order of 15--30~min). The voltage at this point was
recorded as the RMP of the cell. Figure~\ref{fig:I_K_2pore-rmp} shows
the results of such calculations under two different external
concentrations \Ko = 5~mM and \Ko = 25~mM. With increasing blockage of
$I_{\rm K_{2\, pore}}$, the RMP increases. The quantitative values for
this increase match the few experimental data points in
\citet{Clarketal2011}, including the fact that the RMP is higher for
the case where \Ko = 25~mM. Furthermore, the numerical simulations
reveal that this increase with increasing $I_{\rm K_{2\, pore}}$
blockage is relatively linear at the lower \Ko{} value, and starts to
become more nonlinear as \Ko{} is increased.

Finally, the model was used to study the effect of $I_{\rm K_{2\,
    pore}}$-blockage on the volume evolution of the cell. For this,
the osmolarity in the interior and exterior of the cell was related to
the rate of change of volume via (Zhang et al., 1990):
\begin{equation}
  \frac{\mathrm{d}\mathrm{vol}}{\mathrm{d}t} = P_f\;
  \mathrm{area}(t)\; V_W\; (\mathrm{osm}_i(t) - \mathrm{osm}_o)
  \label{dvoldt}
\end{equation}
Here, $P_f$ = 10.2e-4~cm/s is the water permeability of the cell and
V$_W$ = 18.0~cm$^3$/s is the molar volume of water. The surface area
of the cell was related to the evolving volume by assuming a spherical
geometry,
\begin{equation*}
\mathrm{area}(t) = 6^{2/3}\; \pi^{1/3}\;
  \mathrm{vol}(t)^{2/3}.
\end{equation*}
The interior and exterior osmolarities were computed by summing the
(fixed) exterior concentrations of species and (evolving) interior
concentrations of species respectively:
\begin{equation*}
  \begin{split}
  \mathrm{osm}_i &= \left[Na^{+}\right]_{i} + \left[K^{+}\right]_{i} +
  \left[Ca^{2+}\right]_{i} + \left[H^{+}\right]_{i} +
  \left[Cl^{-}\right]_{i}\\
  \mathrm{osm}_o &= \left[Na^{+}\right]_{o} + \left[K^{+}\right]_{o} +
  \left[Ca^{2+}\right]_{o} + \left[H^{+}\right]_{o} +
  \left[Cl^{-}\right]_{o}\\
  \end{split}
\end{equation*}
To ensure that the system does not evolve its volume at initial steady
state, the initial difference between the osmolarities inside and
outside the cell is subtracted from the osmolarity difference in
Equation~\ref{dvoldt}.

Starting from homeostasis, multiple simulations were carried out with
different fractions of $I_{\rm K_{2\, pore}}$ blocked (from 0\% to
100\%). The cellular volume was allowed to evolve over the course of
the simulations as a function of the varying species concentrations,
and the volume at 15~s were
recorded. Figure~\ref{fig:cell_swelling_homeostasis} shows that as a
larger fraction of $I_{\rm K_{2\, pore}}$, the larger the relative
swelling of the cell.

% Local Variables:
% TeX-master: "chondrocyte-model"
% mode: latex
% mode: flyspell
% End:
