\section*{Results}
\label{results}

\todo{{\bf Harish:} Describe the results better}

Our aim in the results is to show:
\begin{itemize}
\item That we have (a) parameterised and (b) validated the model based
  on available experimental data 
\item The characteristics of the model
\item The results of testing our hypothesis, namely that cupivacaine
  blocks $I_2poreK$, leading to changes in the RMP, leading to changes
  in the volume regulation (which concomitantly leads to changes in
  signaling and potentially apoptosis, or changes therein)
\end{itemize}

These two portions of the results can be presented as two main
categories: ``model-revealing figures'' and ``Hypothesis Figures''.

\subsection*{Model characteristics}
In this case, we will present a story which allows us to explain
results of the model and the physiological rationale/relevance of each
component. Simulations for this portion will include both
steady-state, voltage-clamped experiments to show time-independent
behavior, as well as time-dependent currents traces where
applicable. A reasonable voltage range is that used by R. Clark in his
experiments, -130 to +100 mV. An approximate figure list follows.

\begin{itemize}
\item Overall cell behavior: Model schematic. IV curves.
\item Background currents: Input resistance, IV curves, the membrane itself.
\item Pumps and exchanger currents and evolving concentrations: IV
  curves, current traces, concentrations over time, pH versus volume
  (Lewis, et al) -- this shows us how the chondrocyte accounts for
  osmolarity, how we keep track of it.
\item Potassium currents: show all aspects of time-dependent current,
  $I_{kur}$, show main conductances, i.e. $I_{Ca-act K}$ (BK), explain
  basis for RMP
\end{itemize}

\todo{{\bf Harish:} Debug the hypothesis plots and talk about their
  implications}

\subsection*{Bupivicane as a volume regulator in the chondrocyte}
\begin{itemize}
\item Bup. block as initially based on Bob's experiments.
\item 10 percent, 25 percent, 50 percent, 75 percent, 100 percent, and
  perhaps more levels of blockage of $I_{K-2pore}$ conductance
\item Investigate effects on RMP.
\item What adjusts itself in response? Currents? Present these
    currents/concentrations with respect to control.
\item Look specifically at changes in $pH_i$, and link this to volume
  regulation, based on simple formalism from Lewis et al. Creates some
  speculation re: mechanisms that will be interesting and useful for
  Discussion.
\end{itemize}

\todo{{\bf Harish:} List of figures for the results section}

\begin{itemize}
\item IV-curve for the entire cell
\end{itemize}

% Local Variables:
% TeX-master: "chondrocyte-model"
% mode: latex
% mode: flyspell
% End:
