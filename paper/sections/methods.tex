\section*{Model and Methods}
\label{sec:model-and-methods}

We focus our attention on a single chondrocyte cell residing in deep
regions of cartilage. This extracellular environment can be modelled
simply by fixing external concentrations \Nao, \Ko, \Cao, \Ho{} and
\Clo{} within physiologically-relevant ranges (see
Table~\ref{tab:external-concentrations-2}). The chondrocyte cell
membrane boasts a host of voltage- and ligand-gated ion channels as
well as pumps and exchangers \citep{UNKNOWN}; the channels under under
consideration in this model are illustrated in
Figure~\ref{fig:chondrocyte-model} and described in the following
section.

\subsection*{Ionic current formulations}
\label{sec:formulation-ionic-current}

\todo{Needs some introductory text here. Point to the fact that the
  discussion contains other identified channels not explicitly
  modelled in this work.}

\subsubsection*{Potassium channels}
\label{sec:potassium-channels}

Experimental results reported by \citet{Clarketal2011} suggest that
potassium channels play a dominant role in controlling the RMP of the
human tibial joint articular chondrocyte. Motivated by these
observations, our mathematical model incorporates the following three
primary channels for potassium ion transport.

{\bf Two-pore potassium} channels are a set of widely-expressed
K$^{+}$-selective channels whose activation is largely independent of
membrane potential. They appear to play a vital role in determining
the RMP of the cell. Following \citet{UNKNOWN}, the mathematical
expression used to compute this current is:
\begin{equation}
 I_{\rm K_{2\, pore}} = P_{\rm K}\, \frac{z_{\rm K}^2\, V\, F^{2}}{R\,
   T}\, \frac{({\left[K^{+}\right]_{i}} - {\left[K^{+}\right]_{o}}\,
 exp(\frac{-z_{\rm K}\, V\, F}{R\, T}))}{(1 - \exp(-z_K\, V\, F/(R\,
 T))},
\end{equation}
and Figure~\ref{fig:potassium-currents}a shows the voltage-current
curve for this channel fit to experimental data.

Several experimental studies point to the existence of (large) {\bf
  calcium-activated potassium} channels
\citep{BarrettJolleyetal2010}. Such channels are hypothesised to act
as ``osmolytic channels,'' responsible for decreasing intracellular
osmotic potential by fostering efflux of potassium ions. This affects
the ability of the chondrocyte to regulate its volume under rapid
changes in physiochemical environment \citep{Lewisetal2011}. In
addition, studies suggest \citep{UNKNOWN} that this channel can be
stretch-activated (stretch causes an increase in calcium influx, which
results in markedly increased potassium current).

In the present formulation, we ignore the stretch dependence and model
the (large) calcium-activated potassium channel using a functional
form defined by \citet{HorriganAldrich2002}:
\begin{equation}
    I_{\rm K_{Ca-act}} = N_{\rm K_{Ca-act}}\, P_0\, G_{\rm max}\, (V -
    E_{\rm K}),
\end{equation}
where,
\begin{equation}
  \begin{split}
    kTe & = 23.54\, (T/273),\\
    L_v & = L0\, \exp((V\, Z_L)/kTe),\\
    J_v & = \exp(((V - Vh_j)\, Z_j)/Kate),\\
    K & = Ca_i/KDc,\\
    P_0 & = \frac{L_v\, (1+K\, C+J_v\, D+J_v\, K\, C\, D\, E)^4}
    {L_v\, (1+K\, C+J_v\, D+J_v\, K\, C\, D\, E)^4 +
      (1+J_v+K+J_v\, K\, E)^4},\\
    E_{\rm K} & =  \frac{R T}{z_{\rm K} F}
    \ln\left(\frac{\left[K^{+}\right]_{o}}
      {\left[K^{+}\right]_{i}}\right).
  \end{split}
\end{equation}
Figure~\ref{fig:potassium-currents}b shows the voltage-current
curve for this channel fit to experimental data
\citep{Clarketal2011}.

The {\bf delayed-rectifier} was one of the channels found in the
chondrocyte \citep{Walshetal1992, Sugimotoetal1996,
  Mobasherietal2005}. These usually repolarize active cells following
action potentials but their role in chondrocytes are not known because
chondrocytes are far more depolarised. Kv 1.4 and 1.6
\citep{Clarketal2010, Mobasherietal2005} are known to exist. Others
might as well. In this work, the mathematical expression for the
delayed rectifier is motivated by the ultra-rapidly rectifying
potassium channel \citep{Maleckaretal2009}:

\begin{equation}
    I_{\rm K_{\rm ur}} = g_{\rm K_{\rm ur}}\, a_{\rm ur}\, i_{\rm
      ur}\, (V - E_{\rm K}),
\end{equation}
where $i_{\rm ur}$ and $a_{\rm ur}$ are computed as part of the
solution of the ODE system defined by Equation~\ref{ode-system}, and
the following expressions define quantities related to this
time-dependent channel:
\begin{equation}
  \begin{split}
    E_{\rm K} & =  \frac{R T}{z_{\rm K} F}
    \ln\left(\frac{\left[K^{+}\right]_{o}}
      {\left[K^{+}\right]_{i}}\right),\\
    a_{{\rm ur}_{\infty}} & = \frac{1}{1 + \exp(-(V_{\rm m} +
      6.0)/8.6)},\\
    i_{{\rm ur}_{\infty}} & = \frac{1}{1 + \exp(-(V_{\rm m} +
      7.5)/10.0)) + 0.7},\\
    \tau_{a_{\rm ur}} & = \frac{0.009}{1 + \exp((V + 5.0)/12.0)} +
    0.0005,\\
    \tau_{i_{\rm ur}} & = \frac{0.5}{1 + \exp((V +60.0)/20.0)} +
    6.\\
  \end{split}
\end{equation}

\subsubsection*{Pumps and exchangers}
\label{sec:pumps-and-exchangers}

[YES] {\bf Sodium-potassium pump:} Chondrocyte cell volume, as for
other cell types, is determined by a pump-leak model---a double Donnan
equilibrium existing between the intracellular compartment and the
matrix \citep{Stockwell1991}. The effective exclusion of Na+ ions from
the cell is achieved by the activity of the Na+-K+ ATPase, and volume
is maintained by altered balance of leaks and pumps to hold cell water
constant. Due to the high Na+ of their surroundings, chondrocytes are
known to have a high Na+-K+ ATPase activity, with expression and
functional activity upregulated to raised extracellular Na+
\citep{Mobasherietal1997}.

Sodium Potassium Pump \citep[Table 12, pp. 77]{Nygrenetal1998}:
\begin{equation}
  I_{\rm NaK} =
  \bar{I}_{\rm NaK} \left( \frac{[\rm K^{+}]_{\rm o}}{[\rm K^{+}]_{\rm o} +
    k_{\rm NaK_{K}}} \right) \left(\frac{[\rm Na^{+}]^{1.5}_{\rm i}}{[\rm
    Na^{+}]^{1.5}_{\rm i} + k^{1.5}_{\rm NaK_{Na}}}\right) \left( \frac{V + 150}{V +
    200} \right)
\end{equation}

[YES] {\bf Sodium-Calcium exchanger:}
\todo{{\bf Molly:} This channel is a primary calcium concentration
  regulation mechanism. Need to explain the importance of this.}

Sodium Calcium Exchanger \citep[Table 13, pp. 77]{Nygrenetal1998}:
\begin{equation}
  I_{\rm NaCa} = k_{\rm NaCa} \frac{[\rm Na^{+}]^{3}_{i}[\rm
    Ca^{2+}]_{o} \exp(\frac{\gamma V F}{R T}) - [\rm
    Na^{+}]^{3}_{o}[\rm Ca^{2+}]_{i} \exp(\frac{(\gamma - 1.0) V F}{R
      T})} {1.0 + d_{\rm NaCa}([\rm Na^{+}]^{3}_{o}[\rm Ca^{2+}]_{i} +
    [\rm Na^{+}]^{3}_{i}[\rm Ca^{2+}]_{o})}
\end{equation}

[YES] {\bf Some mechanism for sensing external pH}: ASIC channels
(related to ENaC) are opened by extra-cellular protons (in the acidic
environment of the chondrocytes ) and mediate an increase in
Ca2+. There is also some other possible mechanism involving something
called connexin-43. We use the Na+-H+ antiporter from
\citet{Halletal1996,Wilkinsetal2000}.

\todo{{\bf Molly:} Need to explain why we choose the Na-H exchanger
  over a possible ASIC channel}

Sodium Hydrogen Exchanger \citep[Eq. 2, pp. 2675]{Chaetal2009}
\begin{equation}
  \begin{split}
    I_{{\rm NaH}_{\rm mod}} & = \frac{1}{1 + (K_{\rm i}^{n_{\rm
          H}}/[{\rm H}^{+}]_{\rm i}^{n_{\rm H}})}\\
    t_{1} & = \frac{k_{1}^{+} [{\rm Na}^{+}]_{\rm o}/K_{\rm Na}^{\rm
        o}} {(1 + [{\rm Na}^{+}]_{\rm o}/K_{\rm Na}^{\rm o} + [{\rm
        H}^{+}]_{\rm o} /K_{\rm H}^{\rm o})}\\
    t_{2} & = \frac{k_{2}^{+} [{\rm H}^{+}]_{\rm i}/K_{\rm H}^{\rm i}}
    {(1 + [{\rm Na}^{+}]_{\rm i}/K_{\rm Na}^{\rm i} + [{\rm
        H}^{+}]_{\rm i}/K_{\rm H}^{\rm i})}\\
    t_{3} & = \frac{k_{1}^{-} [{\rm Na}^{+}]_{\rm i}/K_{\rm Na}^{\rm
        i}} {(1 + [{\rm Na}^{+}]_{\rm i}/K_{\rm Na}^{\rm i} + [{\rm
        H}^{+}]_{\rm i} /K_{\rm H}^{\rm i})}\\
    t_{4} & = \frac{k_{2}^{-} [{\rm H}^{+}]_{\rm o}/K_{\rm H}^{\rm
        o}} {(1 + [{\rm Na}^{+}]_{\rm o}/K_{\rm Na}^{\rm o} + [{\rm
        H}^{+}]_{\rm o} /K_{\rm H}^{\rm o})}\\
    I_{{\rm NaH}_{\rm exch}} & = \frac{(t_1 t_2 - t_3 t_4)}
    {(t_1 + t_2 + t_3 + t_4)}\\
    I_{\rm NaH} & = N_{\rm NaH} I_{\rm NaH_{\rm mod}}
    I_{\rm NaH_{\rm exch}}
  \end{split}
\end{equation}

\subsubsection*{Background Leakage Currents}
\label{sec:background-currents}

The model accounts for background leakage of \Na{} and \K{} through
the use of time-independent channels whose mathematical expressions
are motivated by Hodgkin and Huxley:

\begin{equation}
  \begin{split}
    I_{\rm Na_b} & = \bar{g}_{\rm Na_b} (V_{\rm m} - E_{\rm Na}),\\
    I_{\rm K_b} & = \bar{g}_{\rm K_b} (V_{\rm m} - E_{\rm K}),\\
  \end{split}
\end{equation}
where the Nernst potentials for the two species are computed in terms
of their respective interior and exterior concentrations:
\begin{equation}
  \begin{split}
    E_{\rm Na} & =  \frac{R T}{z_{\rm Na} F}
    \ln\left(\frac{\left[Na^{+}\right]_{o}}
      {\left[Na^{+}\right]_{i}}\right),\\
    E_{\rm K} & =  \frac{R T}{z_{\rm K} F}
    \ln\left(\frac{\left[K^{+}\right]_{o}}
      {\left[K^{+}\right]_{i}}\right).
  \end{split}
\end{equation}

Analogously, the model accounts for chloride leakage through a similar
mathematical expression,

\begin{equation}
  I_{\rm Cl_b} = \bar{g}_{\rm Cl_b} (V_{\rm m} - E_{\rm Cl}),
\end{equation}
where
\begin{equation}
  E_{\rm Cl} =  \frac{R T}{z_{\rm Cl} F}
  \ln\left(\frac{\left[Cl^{-}\right]_{o}}
          {\left[Cl^{-}\right]_{i}}\right)
\end{equation}
is the Nernst potential set up by the difference in \Cl{}
concentration inside and outside the cell.

While the above background leakage currents are mostly incorporated in
the model as a means of accounting for ion transport not explicitly
modelled by the previously-introduced channels, experimental studies
have managed to isolate one specific chloride leakage channel in human
articular chondrocytes: CFTR \citep{UNKNOWN}. Such channels are likely
necessary for anion loss and may thus be important in regulating the
RMP of the cell.

\subsection*{The atypical environment of the chondrocyte (how
  parameters are chosen)}
\label{sec:chondrocyte-environment}

\todo{{\bf Molly:} Polish the notes below to motivate the choice of
  external concentrations of species.}

\begin{itemize}
  \item The high number of fixed negative charges on the proteoglycan
    (GAGs) attract free cations (e.g. Na+) and exclude free anions from
    the matrix. With cation accumulation, water is osmotically imbibed,
    resulting in lowered pH in comparison with other extracellular
    environments \citep{Wilkinsetal2000, LeeUrban1997}.
  \item Extracellular pH affects the chondrocyte metabolism and its
    ability to synthesise the matrix
    \citep{BarrettJolleyetal2010}. However, rate of collagen
    synthesis seems to be independent of pH \citep{Wuetal2007}.
  \item Due to lack of vascularisation, chondrocytes should scavenge
    precursor molecules for matrix macromolecular synthesis
    \citep{Holmetal1998, Stockwell1991}. Synovial fluid supplies adult
    articular cartilage with (small amounts of) nutrients and oxygen
    (and removes byproducts) by diffusion \citep{LeeUrban1997,
      Otte1991}.
  \item Chondrocytes generate ATP by substrate-level phosphorylation
    during anaerobic respiration which generates H+ ions and further
    lowers surrounding pH \citep{LeeUrban1997}.
  \item Mechanical loading during activity exposes chondrocytes to
    profound fluctuations in their physiochemical environment
    \citep{Mowetal1999, Urban1994}.
    Intracellular concentrations too fluctuate with load. The actual
    biophysics of this is relatively uknown. (We will speculate about
    this along with our modelling.)
\end{itemize}

Some experimentally-reported values for the external concentrations of
these different species from are reported in the following table
\ref{tab:external-concentrations-2}. (We use these in concert with
values from \cite{Clarketal2011}.)

% \begin{table}[ht]
% \begin{centering}
% \begin{tabular}{r c c c c}
% \hline\hline
%              & \Nao (mM) & \Ko (mM) & \Cao (mM) & pH\\
% \hline
% Surface zone & 240--270  & 7--9     & 6--9      & 7.1--7.3\\
% Deep zone    & 300--350  & 9--12    & 14--20    & $\sim$6.9\\
% \hline
% \hline
% \end{tabular}
% \caption{Experimental ranges of external concentrations
%   \citep{Halletal1996}.}
% \label{tab:external-concentrations-1}
% \end{centering}
% \end{table}


\begin{table}[ht]
\begin{centering}
\begin{tabular}{r c c c c}
\hline\hline
             & Cytoplasm & Matrix & Serum/Synovium\\
\hline
\Nao (mM) & 40       & 240--350 & 140\\
\Ko (mM)  & 120--140 & 7--12    & 5\\
\Cao (mM) & 8.e-5 & 6--15 & 1.5\\
$[\mathrm{Cl}^{-}]_{\mathrm{o}} (mM)$ & 60--90 & 60--100 & 140\\
$[\mathrm{HCO^{-}_{3}}]_{\mathrm{o}} (mM)$ & 20 & 15 & 23\\
$[\mathrm{SO^{2-}_{4}}]_{\mathrm{o}} (mM)$ & 0.17 & 0.30 & 0.81\\
pH (mM) & 7.1 & 6.6--6.9 & 7.4\\
Osmolarity (mOsm) & --- & 350--450 & 300\\
\hline
\hline
\end{tabular}
\caption{Experimental ranges of external concentrations
  \citep{Wilkinsetal2000}.}
\label{tab:external-concentrations-2}
\end{centering}
\end{table}

\subsection*{Theoretical model of chondrocyte electrophysiology}
\label{sec:theoretical-model}

In order to simplify the treatment, we assume that there are no
spatial variations in these quantities of interest, allowing us to
model the cell as the following set of ordinary differential equations
(ODEs) in time.

\begin{equation}
  \frac{d}{dt}
  \left(
    \begin{array}{c}
      V_{\rm m}\\
      \left[Na^{+}\right]_{i}\\
      \left[\rm K^{+}\right]_{i}\\
      \left[\rm Ca^{2+}\right]_{i}\\
      \left[\rm H^{+}\right]_{i}\\
      \left[\rm Cl^{-}\right]_{i}\\
      a_{\rm ur}\\
      i_{\rm ur}\\
    \end{array}
  \right)  = \left(
    \begin{array}{c}
        (-I_{i} + I_{\rm stim})/{C_{\rm m}}\\
      - (I_{\rm Na_{b}} + 3\, I_{\rm NaK} + 3\, I_{\rm NaCa} - I_{\rm
        NaH})/(v_{i}\, F)\\
      - (I_{\rm K_{b}} - 2\, I_{\rm NaK} + I_{\rm K_{ur}} + I_{\rm
        K_{2\, pore}} + I_{\rm K_{Ca-act}} + I_{\rm K_{ATP}})/(v_{i}\,
      F)\\
        (I_{\rm NaCa})/(v_{i}\, F)\\
      - (I_{\rm NaH})/(v_{i}\, F)\\
      + (I_{\rm Cl_{b}})/(v_{i}\, F)\\
      (a_{{\rm ur}_{\infty}} - a_{\rm ur})/\tau_{a_{\rm ur}}\\
      (i_{{\rm ur}_{\infty}} - i_{\rm ur})/\tau_{i_{\rm ur}}\\
    \end{array}
  \right)
  \label{ode-system}
\end{equation}

\noindent where,

\begin{equation*}
    \begin{split}
      I_{i} =
      & \phantom{+\,} \underbrace{I_{\rm Na_b} + I_{\rm K_b} + I_{\rm Cl_b}}_{\rm
        Background\, currents}\\
      & +\, \underbrace{I_{\rm NaK} + I_{\rm NaCa} + I_{\rm NaH}}_{\rm
        Pumps\, and\, exchangers}\\
      & +\, \underbrace{I_{\rm K_{ur}} + I_{\rm K_{2\, pore}} + I_{\rm
          K_{Ca-act}} + I_{\rm K_{ATP}}}_{\rm Potassium\, channels}\\
      & +\, \underbrace{I_{\rm VGSC} + I_{\rm TRP_{V4}} + I_{\rm
          stim}}_{\rm Other\, currents}
    \end{split}
\end{equation*}

This ODE system is solved for the primary vector of unknowns: $V_{\rm
m}$, \Nai, \Ki, \Cai, \Hi, \Cli, $a_{\rm ur}$, and $i_{\rm ur}$.

\todo{{\bf Harish:} Discuss details of the solution process here. Talk
  about the methods used \citep{RadhakrishnanHindmarsh1993},
  parametrisation of the model and the fact that the corresponding
  code is available for free for anyone to test and extend.}

\todo{{\bf Harish:} List of figures for the methods section}

\begin{itemize}
\item Overall cell behaviour
\end{itemize}

% Talk about:
% \begin{itemize}
% \item difference wrt to Barett-Jolley 2011 in terms of channels chosen
% \item availability of code
% \item the fact that we are introducing a ``comprehensive model'' but
%   will only focus on some channels
% \item parameters and their estimation
% \end{itemize}

% \begin{sidewaystable}[ht]
% \begin{tabular}{r c l l}
% \hline\hline
% Current description & Notation & Functional form & Parameter values \\ [0.5ex]
% \hline
% Background sodium & $I_{\rm Na_b}$ & $\bar{g}_{\rm Na_b} (V_{\rm m} - E_{\rm Na})$ \cite{UNKNOWN}
%                           & $\bar{g}_{\rm Na_b} = $ \cite{UNKNOWN}, $E_{\rm Na} = $ \cite{UNKNOWN}\\
% Background potassium & $I_{\rm K_b}$ & $\bar{g}_{\rm K_b} (V_{\rm m} - E_{\rm K})$ \cite{UNKNOWN}
%                           & $\bar{g}_{\rm K_b} = $ \cite{UNKNOWN}, $E_{\rm K} = $ \cite{UNKNOWN}\\
% Sodium-potassium pump & $I_{\rm NaK}$ & $\bar{I}_{\rm NaK}
% \frac{[\rm K^{+}]_{\rm c}}{[\rm K^{+}]_{\rm c} + k_{\rm NaK_{K}}}
% \frac{[\rm Na^{+}]^{1.5}_{\rm i}}{[\rm Na^{+}]^{1.5}_{\rm i} + k^{1.5}_{\rm
%     NaK_{Na}}}
% \frac{V + 150}{V + 200}$\cite{Nygrenetal1998} & \cite{Nygrenetal1998}\\
% Sodium-calcium exchanger & $I_{\rm NaCa}$ & $k_{\rm NaCa}
% \frac{[\rm Na^{+}]^{3}_{i}[\rm Ca^{2+}]_{c} \exp(\frac{\gamma V F}{R T}) -
% [\rm Na^{+}]^{3}_{c}[\rm Ca^{2+}]_{i} \exp(\frac{(\gamma - 1.0) V F}{R T})}
% {1.0 + d_{\rm NaCa}([\rm Na^{+}]^{3}_{c}[\rm Ca^{2+}]_{i} + [\rm
%   Na^{+}]^{3}_{i}[\rm Ca^{2+}]_{c})}$
% \cite{Nygrenetal1998} & \cite{Nygrenetal1998}\\
% Sodium-hydrogen exchanger & $I_{\rm NaH}$ & \cite{UNKNOWN} & \cite{UNKNOWN}\\
% Ultra-rapidly rectifying potassium & $I_{\rm K_{ur}}$ & $g_{\rm
%   K_{ur}}\, a_{\rm ur}\, i_{\rm ur}\, (V_{\rm m} - E_{\rm K})$ \cite{Maleckaretal2009} & \cite{Maleckaretal2009}\\
% Two-pore potassium channel & $I_{\rm K_{2\, pore}}$ & \cite{UNKNOWN} & \cite{UNKNOWN}\\
% Calcium-activated potassium & $I_{\rm Ca_{act}K}$ & \cite{UNKNOWN} & \cite{UNKNOWN}\\
% Trip channel(s) & $I_{\rm TRP}$ & $\bar{g}_{\rm NaCa_{TRP}}\, (V_{\rm
%   m} - E_{\rm NaCa})$ \cite{UNKNOWN} & \cite{UNKNOWN}\\
% Applied stimulus & $ I_{\rm stim}$ & Mirroring experiments \cite{Clarketal2011} &  --- \\ [1ex]
% \hline
% \end{tabular}
% \caption{Details of the model}
% \label{tab:chondrocyte-model-details}
% \end{sidewaystable}

% Local Variables:
% TeX-master: "chondrocyte-model"
% mode: latex
% mode: flyspell
% End:
