\section{Hypothesis}
\label{hypothesis}

\begin{itemize}
\item It is known that the progression of osteoarthritis (Rush and
  Hall, 2003) and limited recovery of chondrocytes (Jones et al, 1999)
  is linked to poorly-regulated volume changes (Lewis et al, 2011) and
  physical damage to cartilage is easier in the context of reduced
  osmolarity (Bushet et al, 2005)
\item In turn, there is indication that these volume changes are
  linked to abnormal maintenance of resting membrane potential in
  these cells (Lewis et al, 2011)
\item In healthy cartilage, the resting potential in normal
  chondrocytes $\rightarrow$ carefully regulates proliferation and
  cartilage renewal (cite!)
\item In unhealthy cells, however, response to challenges/stimuli may
  change (e.g. much larger changes in resting membrane potential)
  (Lewis et al, 2011; WIlson, et al 2004; Tsuga 2002; Tirabashi 2010a)
\item It has been suggested that these changes in the regulation of
  the resting membrane potential are due to a variety of remodeled
  channels, including K+, Na+, and Cl- (Lewis et al, 2011; WIlson, et
  al 2004; Tsuga 2002; Tirabashi 2010a)
\item This report concerns itself with the specific case wherein
  buvipocaine, a local anaesthetic, is injected into the shoulder
  capsule to relieve chronic and intractable pain associated with
  joint surgeries and progressive osteoarthritis (cite!)
\item While this administration relieves pain, but can lead to "frozen
  shoulder" syndrome, which may be mediated by cellular apoptosis
  (cite)
\item It is known that the 2pore K+ channel is likely affected by this
  local anaesthetic (cite!)
\item Under what circumstances does this failure of articular
  cartilage and apoptosis occur?
\item We hypothesize that the alteration of the 2pore K+ channel in human articular chondrocytes by local anaesthetic leads to abnormal regulation of the resting membrane potential in these cells, which may concomitantly lead to abnormal volume regulation, altered signaling, and cell death.
\end{itemize}

% Local Variables:
% TeX-master: "chondrocyte-model"
% mode: latex
% mode: flyspell
% End:
