\section{Discussion}
\label{sec:discussion}

\todo{Reiterate what was done; that this was just a start}

\todo{Discuss results from the calculation and the implication}

\subsection{Channels not considered in our model}
\label{sec:channels-ignored}

[NO] {\bf Small calcium-activated Potassium $I_{SK}$}: A few reports
suggest this might exist \citep{Halletal1996,
  BarrettJolleyetal2010}. Tiny magnitude, by definition and role
unknown. So we ignore it.

[NO] {\bf Voltage-gated calcium channels}: T, L-type VGCC found in
some cartilage. Others refute. Supposedly, aggrecan and collagen
synthesis induced by electrical stimulation relies on this channel. We
will claim not important for RMP, just tissue growth, and thus we do
not consider.

[NO] {\bf Epithelial sodium channels (EnaC)}: Not clear what role this
plays in chondrocytes, though it has been identified. People speculate
it has something to do with mechanotransduction $\rightarrow$
contributes to RMP. It is perhaps defective during osteoarthritis. We
will point out we do not look at this as it pertains to mechanics.

[NO] {\bf Aquaporin channels:} AQP or some other mechanism for
transport of water seems super important to the functioning of the
cell. Studies show loss of volume regulation with inhibition of
AQP. But we will point out that we do not model it because it pertains
to mechanics.

[NO] {\bf NMDA channels:} This is an excitatory neuro-transmitter
receptor. It is possibly linked to mechanotransduction.

\todo{Use these missing channels to motivate next steps}

Uncited references:
\cite{ArcherWest2003},
\cite{Grishkoetal2010},
\cite{Hille2001},
\cite{LesageLazdunski2000},
\cite{MillwardSadleretal2000},
\cite{Nygrenetal1998},
\cite{RadhakrishnanHindmarsh1993},
\cite{Scholz2002},
\cite{Tsugaetal2001}.

% Local Variables:
% TeX-master: "chondrocyte-model"
% mode: latex
% mode: flyspell
% End:
