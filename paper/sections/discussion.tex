\section*{Discussion}
\label{sec:discussion}

\todo{Discuss results from the calculation and their implications}

\subsection*{Limitations}
\label{sec:limitations}

\todo{Use these missing channels to motivate subsequent modelling
  steps}

{\bf TRP channels (TRPv4 in particular)}: These show little
selectivity between cations Na+, Ca2+, Mg2+. Several have been
identified in osteo-arthritic cartilage. Knocking out TRPv4
$\rightarrow$ loss of Ca2+ response to hypo-osmotic challenge
$\rightarrow$ osteoarthritic changes. (Will cite.) Thus, membrane
stretch $\rightarrow$ calcium mobilisation via TRPv4 $\rightarrow$
activation of Ca-act K, chondrogenesis and volume regulation.

Osteo-arthritic trip channel \citep{UNKNOWN}:
\begin{equation}
  I_{\rm TRP_{V4}} = \bar{g}_{\rm TRP_{V4}}\, (V_{\rm m} - E_{\rm ?})
\end{equation}

{\bf Voltage-gated sodium channels}: These have been found in
some rabbit chondrocytes and some osteo-arthritic cartilage. They are
likely constantly inactivated due to the voltages the
chondrocyte.  We instead model $I_{Na_{b}}$ as a sort of constant
background Na+ leak.

External stimulation (matching experiments, e.g. cyclic stimulation):
\begin{equation}
I_{\rm stim} = \overline{I_{\rm stim}}\, {\rm square}(\frac{2 \pi\,
  t}{t_{\rm cycle}}, \frac{t_{\rm stim}}{t_{\rm cycle}})
\end{equation}

{\bf Inwardly-rectifying $I_{K_{ATP}}$}: This channel is closed by
intracellular ATP. It allows for physical coupling between metabolism
and membrane excitability. It is observed to be open in the low-oxygen
tension environment of chondrocytes \citep{DartStanden1994,
  Mobasherietal2007}.

{\bf Some mechanism for sensing external pH}: ASIC channels
(related to ENaC) are opened by extra-cellular protons (in the acidic
environment of the chondrocytes ) and mediate an increase in
Ca2+. There is also some other possible mechanism involving something
called connexin-43.

{\bf Small calcium-activated Potassium $I_{SK}$}: A few reports
suggest this might exist \citep{Halletal1996,
  BarrettJolleyetal2010}. Tiny magnitude, by definition and role
unknown. So we ignore it.

{\bf Voltage-gated calcium channels}: T, L-type VGCC found in
some cartilage. Others refute. Supposedly, aggrecan and collagen
synthesis induced by electrical stimulation relies on this channel. We
will claim not important for RMP, just tissue growth, and thus we do
not consider.

{\bf Epithelial sodium channels (EnaC)}: Not clear what role this
plays in chondrocytes, though it has been identified. People speculate
it has something to do with mechanotransduction $\rightarrow$
contributes to RMP. It is perhaps defective during osteoarthritis. We
will point out we do not look at this as it pertains to mechanics.

{\bf Aquaporin channels:} AQP or some other mechanism for
transport of water seems super important to the functioning of the
cell. Studies show loss of volume regulation with inhibition of
AQP. But we will point out that we do not model it because it pertains
to mechanics.

{\bf NMDA channels:} This is an excitatory neuro-transmitter
receptor. It is possibly linked to mechanotransduction.

%% Uncited references:
%% \cite{ArcherWest2003},
%% \cite{Grishkoetal2010},
%% \cite{Hille2001},
%% \cite{LesageLazdunski2000},
%% \cite{MillwardSadleretal2000},
%% \cite{Nygrenetal1998},
%% \cite{Scholz2002},
%% \cite{Tsugaetal2001}.

% Local Variables:
% TeX-master: "chondrocyte-model"
% mode: latex
% mode: flyspell
% End:
