\pagestyle{myheadings}
\markright{Computational model of chondrocyte electrophysiology}

\maketitle

\abstract{We present a computational model for studying the
  electrophysiology of the human articular chondrocyte. It is
  novel. We present some numerical results to help demonstrate
  aspects of the behaviour of the model. We pay particular attention
  to the potassium channels and the effect of blocking $I_{\rm
    K_{2\; pore}}$ on the cell's RMP. This will serve as a useful
  tool in helping us understand the causes of osteoarthritis and
  things like the ``frozen shoulder'' syndrome.

  \emph{Key words:} chondrocyte; electrophysiology; potassium
  channels; computational model}

\todo{{\bf Harish:} Rewrite completely.

For BioPhys J, it takes the following form:

\begin{itemize}
\item One general intro sentence, setting up
\item Problem statement
\item What we are going to do about it (methods)
\item What happened (results) (Most important)
\item Conclusion
\end{itemize}
}

% Local Variables:
% TeX-master: "chondrocyte-model"
% mode: latex
% mode: flyspell
% End:
