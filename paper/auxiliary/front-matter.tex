\begin{frontmatter}

\title{The role of ${\rm K}^{+}$ channels in human articular
  chondrocyte electrophysiology: a computational perspective}

\author[srl]{Harish~Narayanan\corref{cor}}
\cortext[cor]{Corresponding Author}
\ead{harish@simula.no}
\author[srl]{Mary~M.~Maleckar}
\ead{mmaleck@simula.no}
\author[ucal]{Wayne~R.~Giles}
\ead{wgiles@ucalgary.ca}

\address[srl]{Center for Biomedical Computing, Simula Research
  Laboratory, P.O.~Box~134, 1325~Lysaker, Norway}

\address[ucal]{Faculty of Kinesiology, University of Calgary, Calgary,
  Alberta, Canada T2N 4N1}

\begin{abstract}
  \todo{We present a computational model for studying the
    electrophysiology of the human articular chondrocyte. It is
    novel. We present some numerical results to help demonstrate
    aspects of the behaviour of the model. We pay particular attention
    to the potassium channels and the effect of blocking $I_{\rm
      K_{2\; pore}}$ on the cell's RMP. This will serve as a useful
    tool in helping us understand the causes of osteoarthritis and
    things like the ``frozen shoulder'' syndrome.}
\end{abstract}

\begin{keyword}
chondrocyte \sep electrophysiology \sep potassium channels \sep
computational model
\end{keyword}

\end{frontmatter}

% Local Variables:
% TeX-master: "chondrocyte-model"
% mode: latex
% mode: flyspell
% End:
