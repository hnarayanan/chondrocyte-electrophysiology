\documentclass{article}
\usepackage[a4paper,landscape]{geometry}
\usepackage{color}
\definecolor{missing}{rgb}{1,0,0}

\begin{document}
\centering

{\Large \bf The chondrocyte electrophysiological model\\}

\large
\pagestyle{empty}
\begin{displaymath}
-C_{\rm m} \frac{dV_{\rm m}}{dt} = I_{\rm Na_b} + I_{\rm K_b}
                               + I_{\rm NaK} + I_{\rm NaCa}
                               + \textcolor{missing}{I_{\rm NaH}} + I_{\rm K_{ur}}
                               + \textcolor{missing}{I_{\rm K_{2\, pore}}}
                               + \textcolor{missing}{I_{\rm Ca_{act}K}}
                               + I_{\rm TRP} + I_{\rm stim}
\end{displaymath}

\begin{table}[ht]
\large
\centering
\begin{tabular}{r c l l}
\hline\hline
Current description & Notation & Functional form & Parameter values \\ [0.5ex]
\hline
Background sodium & $I_{\rm Na_b}$ & $\bar{g}_{\rm Na_b} (V_{\rm m} - E_{\rm Na})$ \cite{UNKNOWN}
                          & $\bar{g}_{\rm Na_b} = $ \cite{UNKNOWN}, $E_{\rm Na} = $ \cite{UNKNOWN}\\
Background potassium & $I_{\rm K_b}$ & $\bar{g}_{\rm K_b} (V_{\rm m} - E_{\rm K})$ \cite{UNKNOWN}
                          & $\bar{g}_{\rm K_b} = $ \cite{UNKNOWN}, $E_{\rm K} = $ \cite{UNKNOWN}\\
Sodium-potassium pump & $I_{\rm NaK}$ & $\bar{I}_{\rm NaK}
\frac{[\rm K^{+}]_{\rm c}}{[\rm K^{+}]_{\rm c} + k_{\rm NaK_{K}}}
\frac{[\rm Na^{+}]^{1.5}_{\rm i}}{[\rm Na^{+}]^{1.5}_{\rm i} + k^{1.5}_{\rm
    NaK_{Na}}}
\frac{V + 150}{V + 200}$\cite{Nygrenetal1998} & \cite{Nygrenetal1998}\\
Sodium-calcium exchanger & $I_{\rm NaCa}$ & $k_{\rm NaCa}
\frac{[\rm Na^{+}]^{3}_{i}[\rm Ca^{2+}]_{c} \exp(\frac{\gamma V F}{R T}) -
[\rm Na^{+}]^{3}_{c}[\rm Ca^{2+}]_{i} \exp(\frac{(\gamma - 1.0) V F}{R T})}
{1.0 - d_{\rm NaCa}([\rm Na^{+}]^{3}_{c}[\rm Ca^{2+}]_{i} + [\rm
  Na^{+}]^{3}_{i}[\rm Ca^{2+}]_{c})}$
\cite{Nygrenetal1998} & \cite{Nygrenetal1998}\\
Sodium-hydrogen exchanger & $I_{\rm NaH}$ & \cite{UNKNOWN} & \cite{UNKNOWN}\\
Ultra-rapidly rectifying potassium & $I_{\rm K_{ur}}$ & $g_{\rm
  K_{ur}}\, a_{\rm ur}\, i_{\rm ur}\, (V_{\rm m} - E_{\rm K})$ \cite{Maleckaretal2009} & \cite{Maleckaretal2009}\\
Two-pore potassium channel & $I_{\rm K_{2\, pore}}$ & \cite{UNKNOWN} & \cite{UNKNOWN}\\
Calcium-activated potassium & $I_{\rm Ca_{act}K}$ & \cite{UNKNOWN} & \cite{UNKNOWN}\\
Trip channel(s) & $I_{\rm TRP}$ & $\bar{g}_{\rm NaCa_{TRP}}\, (V_{\rm
  m} - E_{\rm NaCa})$ \cite{UNKNOWN} & \cite{UNKNOWN}\\
Applied stimulus & $ I_{\rm stim}$ & Mirroring experiments \cite{Clarketal2011} &  --- \\ [1ex]
\hline
\end{tabular}
\caption{Details of the model}
\label{table:chondrocyte-model-details}
\end{table}

\normalsize

\begin{thebibliography}{1}

% Hodgkin-Huxley for rationale on the background currents

\bibitem{UNKNOWN}
\emph{Unknown article}, U. Author, Unknown journal, vol.~xx(y), pp.~mm--nn, yyyy.

\bibitem{Maleckaretal2009}
\emph{K$^{+}$ current changes account for the rate dependence of the
  action potential in the human atrial myocyte}, M. M. Maleckar,
J. L. Greenstein, W. R. Giles and N. A. Trayanova, Am J Physiol Heart
Circ Physiol, vol.~297, pp.1398--1410, 2009.

\bibitem{Nygrenetal1998}
\emph{Mathematical Model of an Adult Human Atrial Cell: The Role of
  K+ Currents in Repolarization}, A. Nygren, C. Fiset, L. Firek,
J. W. Clark, D. S. Lindblad, R. B. Clark and W. R. Giles, Circ. Res.,
vol.~82, pp.~63--81, 1998.

\bibitem{Clarketal2011}
\emph{Two-pore K+ channels contribute to membrane potential of
  isolated human articular chondrocytes}, R. B. Clark, C. Kondo and
W. R. Giles, Unknown journal, vol.~xx(y), pp.~mm--nn, yyyy.

\end{thebibliography}

\end{document}