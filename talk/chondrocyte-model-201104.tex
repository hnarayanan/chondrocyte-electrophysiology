% I have provided the source and required media for this presentation
% under the hope that it will be useful. Please don't use it for
% anything nefarious.

% Also, note that this LaTeX file requires the beamer class to
% compile. If you don't already have it, you are expected to obtain
% and install it from http://latex-beamer.sourceforge.net/

% (c) 2011 -- Harish Narayanan

\documentclass[ignorenonframetext]{beamer}

\usepackage{graphicx}
\usepackage{amsfonts}
\usepackage{amssymb}
\usepackage{fancyvrb}
\usepackage{color}
\usepackage{multimedia}
\usepackage[english]{babel}
\usepackage[latin1]{inputenc}
\usepackage{textcomp}

% Custom commands

\newcommand{\references}[1] {
  \begin{flushright}
    \scriptsize [#1] \normalsize
  \end{flushright}
}

\newcommand{\addfigure} {
  \scriptsize
  \textcolor{red}{[insert figure]}
  \normalsize
}

% All sorts of things pertinent to styling beamer
\usetheme{boxes}
\setbeamertemplate{navigation symbols}{}
\usecolortheme{seagull}
\usefonttheme{professionalfonts}
\useinnertheme{circles}
\setbeamercolor{frametitle}{fg=red}
\setbeamercolor{title}{fg=red}

\mode<presentation> {
  \setbeamercovered{transparent}
}

\title{Electrophysiological modelling of chondrocytes in human
  articular cartilage}
\author{H.~Narayanan}
\subject{}
\institute[]{}
\date[]{}

\begin{document}

% 0. Title slide and mapping
%
%   Electrophysiological modelling of chondrocytes in human articular
%   cartilage
%
%   - Introducing the chondrocyte and why it is interesting in the
%     context of cartilage
%   - Some particular scientific questions
%
%   - Model and motivation
%   - Preliminary results
%   - Discussion

\frame{\titlepage}

% 1. What is the chondrocyte and why is it interesting?

\begin{frame}{Chondrocytes are the resident cells of articular
    cartilage and are responsible for maintaining the extracellular
    matrix}

  \begin{itemize}
  \item A brief description of articular cartilage \addfigure
    \begin{itemize}
    \item physiology --  cartilage contains {\em chondrocytes}, ECM,
      proteoglycans
    \item bioengineering -- relatively acellular, avascular; easy to
      study and generate in the lab
    \end{itemize}
  \item A brief description of the chondrocyte \addfigure
    \begin{itemize}
    \item Low cell to tissue volume ratio (few \%), but can do a lot
    \item Signalling (pH sensitivity, ionic content, mechano-sensing
      (stress))---causes matrix synthesis/degradation
    \item Something in general about disruption?
    \end{itemize}
  \end{itemize}

\references{Poole, 1997; Archer \& Francis-West 2002}

\end{frame}

% A lot can happen, but we are interested in something particular
%
% 2. The specific question(s) we want to answer

\begin{frame}{We narrow our focus to specific questions to motivate
    our initial modelling efforts}

  \begin{itemize}
  \item Hypothesis 1: ``Frozen shoulder syndrome'' when tissue is
    exposed to buvipicane (and its links to $I_{K_{2-pore}}$)
  \item Hypothesis 2: pH and volume regulation (via ASIC)
  \item Hypothesis 3: Apoptosis and osteoarthritis?
  \end{itemize}

\references{citation 1; Lewis et al., 2011; citation 3}

\end{frame}

% All of these need a good understanding of important channels
% => basis of resting membrane potential => volume regulation etc.
%
% There is a dearth of good biophysical models for this cell type and
% so we construct our own.
%
% Return to review papers. As of 1996, it looked like [figure], and as
% of 2011 people feel it looks like [figure], so on the whiteboard in
% my office you have a superset of these studies.
%
% 3. Modelling the electrophysiology of a single cell

\begin{frame}{What physiology literature tells us about the channels\\
    in chondrocytes}

  \begin{itemize}
  \item A superset of what is known thus far about the channels in the
    chondrocyte \addfigure
  \item Talk about some important ones and what their roles might be

    \references{Hall et al., 1996; Barrett-Jolley et al., 2010}

  \end{itemize}

\end{frame}

% And so, this is the model that is being researched and
% implemented. Removing the things that are not pertinent to humans
% etc., we end up with the following initial model.
%
% 4. The computational model

\begin{frame}{The electrophysiological model of a single cell}

  \begin{itemize}
  \item All channels we have decided to consider \addfigure %CellML
  \item Return to one specific hypothesis and point out the channels
    that seem important % Modify figure?
  \item Discuss some functional forms, including rationale
  \item Talk about the programming ideas
    \begin{itemize}
      \item ODE Solver
      \item Code layout
      \item Parameter estimation; genetic algorithm idea?
    \end{itemize}
  \end{itemize}

  \references{functional forms, Clark et. al, parameter estimation}

\end{frame}

% And with these parameters, I observe the following behaviour in my
% test cases
%
% 5. Some numerical simulations and their implications

\begin{frame}{What do the numerical simulations tell us?}

  \includegraphics[width=0.8\textwidth]{../results/pdf/membrane_behaviour}

  \begin{itemize}
  \item Attempts at determining parameters; tie with experimental data
  \item Basic benchmarking of the model; Testing our hypothesis?
  \end{itemize}

  \references{parameter estimation, benchmarking, hypothesis}

\end{frame}

% 6. Discussion

\begin{frame}{In conclusion, \ldots}

  \begin{itemize}
  \item Summarise what has been done
    \begin{itemize}
      \item Learnt a bit about the physiology of the chondrocyte
      \item Some ideas of the channels involved
      \item Arrived at some interesting questions
      \item A basic model to answer these questions
    \end{itemize}
  \item For the future
    \begin{itemize}
    \item Ask for modelling advice
    \item Implications of the calculations
    \item Tissue level models and stress response
    \end{itemize}
  \end{itemize}

  \references{code, paper}

\end{frame}

\end{document}

% Local Variables:
% mode: latex
% mode: flyspell
% mode: auto-fill
% End: